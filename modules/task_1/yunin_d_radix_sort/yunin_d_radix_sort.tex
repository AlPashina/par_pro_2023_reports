\documentclass{article}
\usepackage[utf8]{inputenc}
\usepackage[T2A]{fontenc}
\usepackage[russian]{babel}
\usepackage{amsmath}
\usepackage{graphicx}

\date{}
\title{}
\author{}

\begin{document}
\begin{titlepage}
\begin{center}
    {\bfseries МИНИСТЕРСТВО НАУКИ И ВЫСШЕГО ОБРАЗОВАНИЯ \\
        РОССИЙСКОЙ ФЕДЕРАЦИИ}\\Федеральное государственное автономное образовательное учреждение высшего образования\\
    {\bfseries «Национальный исследовательский Нижегородский государственный университет им. Н.И. Лобачевского»\\(ННГУ)\\Институт информационных технологий, математики и механики} \\
\end{center}
\vspace{8em}
\begin{center}
    ОТЧЁТ\\ по лабораторной работе \\«Поразрядная сортировка для вещественных чисел (тип double) с простым слиянием»
\end{center}
\vspace{5em}
\begin{flushright}
    {\bfseries Выполнил:} студент группы 382006-2\\Юнин Д.Д.\underline{\hspace{3cm}} \linebreak\linebreak\linebreak
    {\bfseries Проверил:} младший научный сотрудник\\Нестеров А.Ю. \underline{\hspace{3cm}} 
\end{flushright}
\vspace{\fill}
\begin{center}
    Нижний Новгород\\2023
\end{center}
\end{titlepage}
% Содержание
\tableofcontents
\thispagestyle{empty}
\newpage

\pagestyle{plain}
\setcounter{page}{3}
\maketitle

\section{Введение}

Сортировка данных является одной из ключевых операций в области обработки и анализа информации. В современном мире, где объемы данных растут в геометрической прогрессии, умение эффективно сортировать большие массивы данных является неотъемлемой частью множества задач и приложений. От электронной коммерции до научных исследований, от финансовых операций до социальных сетей, сортировка данных играет важную роль в обеспечении оперативного и точного анализа информации.

Целью данной лабораторной работы является изучение и сравнение разных параллельных реализаций алгоритма поразрядной сортировки (для чисел типа double) с использованием разных технологий. Параллельная обработка данных — это способ ускорения выполнения вычислительных задач за счет распределения работы между несколькими исполнителями (процессорами, ядрами, потоками и т.д.), которые могут выполняться одновременно и независимо друг от друга.

Параллельная обработка данных для поразрядной сортировки заключается в том, что исходный массив разбивается на несколько подмассивов, которые сортируются независимо друг от друга на разных потоках. Затем эти подмассивы объединяются в один массив с помощью простого слияния и как итог, получается, полностью отсортированный массив.

\newpage
\section{Постановка задачи}
В рамках данной работы нужно выполнить следующие задачи:
\begin{enumerate}
    \item Реализовать различные версии алгоритма поразрядной сортировки (типа double) с простым слиянием:  \begin{itemize} \item Последовательная реализация. Алгоритм выполняется в одном потоке. Данная реализация существует для сравнения производительности и корректности работы параллельных реализаций. \item Параллельная реализация алгоритма с применением технологии OpenMP. OpenMP - библиотека, которая позволяет писать параллельные программы на C, C++, Fortran. \item Параллельная реализация алгоритма с применением Intel Threading Building Blocks (TBB). TBB — это библиотека для написания параллельных программ на языке C++. \item Параллельная реализация алгоритма с применением std::thread. std::thread — это класс из стандартной библиотеки C++, который представляет собой низкоуровневый интерфейс для работы с потоками исполнения. \end{itemize}
    \item Сравнить время работы разных реализаций алгоритма на различных наборах данных (массив случайных чисел, упорядоченный массив и т.д.).
    \item Сделать выводы по полученным данным.
\end{enumerate}

\newpage
\section{Описание программной реализации}

\subsection{Функция printVector} 
Функция принимает в качестве параметра вектор и печатает его содержимое на экран. Функция возращает void.

\subsection{Функция getRandomVector} 
Функция принимает в качестве параметров: размер вектора, левую и правую границы для чисел. В ней создаётся вектор конкретного размера и заполяется числа типа double в интвервале от левой границы до правой границы. Функция возвращает созданный вектор.

\subsection{Функция countSort} 
Функция реализует алгоритм сортировки подсчётом, что требуется для поразрядной сортировки. На вход функции в качестве параметров поступают 2 указателя типа double (входной и выходной массивы) и длина (размер массивов).
Функция возвращает void.

\subsection{Функция countSortFinalStep} 
Функция реализует алгоритм сортировки подсчётом с некоторыми модернизациями для последнего шага в поразрядной сортировки. На вход функции в качестве параметров поступают 2 указателя типа double (входной и выходной массивы) и длина (размер массивов). функция возвращает значение типа bool.

\subsection{Функция merge}
Функция реализует алгоритм простого слияния двух отсортированных векторов в один тоже отсортированный вектор.
Создаётся вектор, размер которого равен сумме размеров двух входных векторов.
После чего в цикле for, который продолжается до достижения значения счётчика значения размера созданного вектора, объединяем массивы, сравнивания элементы каждого между собой и добавляем в результирующий вектор элемент того вектора, значения которого меньше и увеличиваем индекс того вектора, чьё значение добавил в результирующий массив. Если мы пришли к концу одного из векторов, то просто записываем в результирующий массив оставшиеся элементы другого вектора.


\subsection{Функция splitVector} 
Функция реализует разделение вектора на подвекторы, и сохраняет их как элементы в результирующий вектор. В качестве параметров функции приходят: массив с данными, количество частей, на которое нужно поделить вектор, временный вектор, который мы будем записывать в результирующий вектор и результирующий вектор, который внутри себя хранит вектора (все вектора хранят в себе значения типа double).

\newpage
\section{Описание алгоритмов}

\subsection{Последовательная версия}

В начале вектор данных разделяется на части.\\Каждая часть сортируется при помощи алгоритма поразрядной сортировки.\\Затем отсортированные части объединяются с помощью функции merge, которая реализует алгоритм простого слияния, в результате которого две отсортированные части объединяются в одну.

\subsection{OpenMP версия}

В первую очередь с помощью функции omp\_set\_num\_threads устанавливается количество потоков, которые будут использованы для параллельной обработки. В данном случае это количество равно количеству частей, на которые разделён вектор данных.

Используется цикл parallel for, который запускает функцию radixSortSeq для каждого элемента вектора параллельно. Функция radixSortSeq выполняет последовательную поразрядную сортировку для каждого подвектора.\\
После чего все части сливаются в один вектор.

\subsection{TBB версия}
Изначально создаётся объект класса tbb::global\_control, который устанавливает максимальное количество потоков, которые могут быть использованы для параллельной обработки. В данном случае это количество равно количеству частей, на которые разделён вектор данных.

blocked\_range - это класс шаблонов из библиотеки TBB, который представляет диапазон значений, который может быть разделен между несколькими потоками. Он используется для разделения итераций цикла между потоками в параллельных вычислениях.

В нашем случае tbb::blocked\_range<int>(0, vectorsForParallel.size()) создает объект blocked\_range, который представляет диапазон значений от 0 до размера вектора vectorsForParallel (равному количеству частей, на которые разделён изначальный вектор данных).

Таким образом, запускается цикл parallel\_for, который разбивает вектор на блоки и запускает функцию radixSortSeq для каждого блока параллельно.

\subsection{std::thread версия}
В этой версии создаётся несколько потоков. Каждый поток выполняет поразрядную сортировку для своего элемента вектора. После того, как все потоки завершат свою работу, они объединяются обратно в главный поток с помощью функции join().

\newpage
\section{Результаты экспериментов}

\noindent\textbf{Результаты измерения времени работы в миллисекундах на 4 потоках:}\\\\
    \begin{tabular}{|c | c | c | c | c |} 
      \hline
      Технология / Размер входных данных & $10^5$ & $10^6$ & $10^7$ & $10^8$ \\
      \hline
      Seq & 2.15 & 21  & 304 & 3111 \\ 
      \hline
      OpenMP & 1.02 & 6.6  & 119 & 1266 \\ 
      \hline
      TBB & 1.1 & 7  & 119 & 1275 \\ 
      \hline
      std::thread & 1.32 & 7.5  & 120 & 1285 \\ 
      \hline
    \end{tabular}


\newpage
\section{Вывод}
В ходе выполнения лабораторной работы были реализованы 4 варианта алгоритма пораздрядной сортировки (последовательная и параллельные версии) на языке программирования C++ и проведены эксперименты, которые показали, что время выполнения всех трех параллельных реализаций примерно равно.

В зависимости от целей и условий задачи можно выбрать различную технологию для параллельной реализации алгоритма поразрядной сортировки. Если требуется максимальная производительность и гибкость, то можно использовать TBB или std::thread. Если требуется простота и удобство использования, то можно использовать OpenMP. Если требуется минимальное время разработки и отладки, то можно использовать последовательную реализацию.

\newpage
\begin{thebibliography}{9} \bibitem{shell-1} Поразрядная сортировка.
https://algolist.manual.ru/sort/radix\_sort.php
\bibitem{shell-2} Radix Sort - GeeksforGeeks. https://www.geeksforgeeks.org/radix-sort/ \end{thebibliography}

\newpage
\section{Приложение}
\begin{verbatim}
void printVector(const std::vector<double>& vec) {
    for (int i = 0; i < vec.size(); i++) {
        std::cout << vec[i] << " ";
    }
    std::cout << std::endl;
}

std::vector<double> randomVec(int size, int leftBound, int rightBound) {
    std::vector<double> vec(size);
    std::mt19937 gen;
    gen.seed(static_cast<uint32_t>(time(0)));
    std::uniform_real_distribution<double> buf(leftBound, rightBound);
    std::transform(vec.begin(), vec.end(), vec.begin(),
        [&](double x) { return buf(gen); });
    return vec;
}

std::vector<double> merge(const std::vector<double>& arr1,
    const std::vector<double>& arr2) {
    int len1 = arr1.size();
    int len2 = arr2.size();
    std::vector<double> out;
    out.resize(len1 + len2);
    int indexFirst = 0, indexSecond = 0;
    for (int i = 0; i < len1 + len2; i++) {
        if (indexFirst == len1) {
            out[i] = arr2[indexSecond];
            indexSecond++;
            continue;
        }
        if (indexSecond == len2) {
            out[i] = arr1[indexFirst];
            indexFirst++;
            continue;
        }
        if (arr1[indexFirst] > arr2[indexSecond]) {
            out[i] = arr2[indexSecond];
            indexSecond++;
        } else if (arr1[indexFirst] <= arr2[indexSecond]) {
            out[i] = arr1[indexFirst];
            indexFirst++;
        }
    }
    return out;
}

void countSort(double* in, double* out, int len, int exp) {
    unsigned char* buf = reinterpret_cast<unsigned char*>(in);
    int count[256] = { 0 };
    for (int i = 0; i < len; i++) {
        count[buf[8 * i + exp]]++;
    }
    /*for (int i = 0; i < 256; i++) {
        std::cout << count[i] << " ";
    }
    std::cout << std::endl;*/
    int sum = 0;
    for (int i = 0; i < 256; i++) {
        int temp = count[i];
        count[i] = sum;
        sum += temp;
    }
    /*for (int i = 0; i < 256; i++) {
        std::cout << count[i] << " ";
    }
    std::cout << std::endl;*/
    for (int i = 0; i < len; i++) {
        out[count[buf[8 * i + exp]]] = in[i];
        count[buf[8 * i + exp]]++;
    }
}

bool countSortFinalStep(double* in, double* out, int len) {
    int firstNegativeIndex = -1;
    int firstPositiveIndex = -1;
    bool positiveFlag = false, negativeFlag = false;
    for (int i = 0; i < len; i++) {
        if (positiveFlag && negativeFlag) {
            break;
        }
        if (in[i] < 0 && !negativeFlag) {
            firstNegativeIndex = i;
            negativeFlag = true;
        }
        if (in[i] > 0 && !positiveFlag) {
            firstPositiveIndex = i;
            positiveFlag = true;
        }
    }
    if (positiveFlag && negativeFlag) {
        int j = len - 1;
        bool flag = false;
        for (int i = 0; i < len; i++) {
            out[i] = in[j];
            if (flag) {
                j++;
            } else {
                j--;
            }
            if (j == firstNegativeIndex - 1 && !flag) {
                j = 0;
                flag = true;
            }
        }
        return true;
    } else if (!positiveFlag) {
        for (int i = len - 1, j = 0; i >= 0; i--, j++) {
            out[j] = in[i];
        }
        return true;
    }
    return false;
}

std::vector<double> radixSort(const std::vector<double>& data1,
    const std::vector<double>& data2) {
    int len1 = static_cast<int>(data1.size());
    int len2 = static_cast<int>(data2.size());
    std::vector<double> in1 = data1;
    std::vector<double> out1(data1.size());
    std::vector<double> in2 = data2;
    std::vector<double> out2(data2.size());
    for (int i = 0; i < 4; i++) {
        countSort(in1.data(), out1.data(), len1, 2 * i);
        countSort(out1.data(), in1.data(), len1, 2 * i + 1);
    }
    if (!countSortFinalStep(in1.data(), out1.data(), len1)) {
        in1.swap(out1);
    }
    for (int i = 0; i < 4; i++) {
        countSort(in2.data(), out2.data(), len2, 2 * i);
        countSort(out2.data(), in2.data(), len2, 2 * i + 1);
    }
    if (!countSortFinalStep(in2.data(), out2.data(), len2)) {
        in2.swap(out2);
    }
    return merge(out1, out2);
}

void printVector(const std::vector<double>& vec) {
    for (int i = 0; i < vec.size(); i++) {
        std::cout << vec[i] << " ";
    }
    std::cout << std::endl;
}

std::vector<double> getRandomVector(int  size, int leftBound, int rightBound) {
    std::vector<double> vec(size);
    std::random_device dev;
    std::mt19937 gen(dev());
    std::uniform_real_distribution<double> buf(leftBound, rightBound);
    for (int i = 0; i < size; i++) {
        vec[i] = buf(gen);
    }
    return vec;
}

std::vector<double> merge(const std::vector<double>& arr1,
    const std::vector<double>& arr2) {
    int len1 = arr1.size();
    int len2 = arr2.size();
    std::vector<double> out;
    out.resize(len1 + len2);
    int indexFirst = 0, indexSecond = 0;
    for (int i = 0; i < len1 + len2; i++) {
        if (indexFirst == len1) {
            out[i] = arr2[indexSecond];
            indexSecond++;
            continue;
        }
        if (indexSecond == len2) {
            out[i] = arr1[indexFirst];
            indexFirst++;
            continue;
        }
        if (arr1[indexFirst] > arr2[indexSecond]) {
            out[i] = arr2[indexSecond];
            indexSecond++;
        } else if (arr1[indexFirst] <= arr2[indexSecond]) {
            out[i] = arr1[indexFirst];
            indexFirst++;
        }
    }
    return out;
}

void countSort(double* in, double* out, int len, int exp) {
    unsigned char* buf = reinterpret_cast<unsigned char*>(in);
    int count[256] = { 0 };
    for (int i = 0; i < len; i++) {
        count[buf[8 * i + exp]]++;
    }
    /*for (int i = 0; i < 256; i++) {
        std::cout << count[i] << " ";
    }
    std::cout << std::endl;*/
    int sum = 0;
    for (int i = 0; i < 256; i++) {
        int temp = count[i];
        count[i] = sum;
        sum += temp;
    }
    /*for (int i = 0; i < 256; i++) {
        std::cout << count[i] << " ";
    }
    std::cout << std::endl;*/
    for (int i = 0; i < len; i++) {
        out[count[buf[8 * i + exp]]] = in[i];
        count[buf[8 * i + exp]]++;
    }
}

bool countSortFinalStep(double* in, double* out, int len) {
    int firstNegativeIndex = -1;
    int firstPositiveIndex = -1;
    bool positiveFlag = false, negativeFlag = false;
    for (int i = 0; i < len; i++) {
        if (positiveFlag && negativeFlag) {
            break;
        }
        if (in[i] < 0 && !negativeFlag) {
            firstNegativeIndex = i;
            negativeFlag = true;
        }
        if (in[i] > 0 && !positiveFlag) {
            firstPositiveIndex = i;
            positiveFlag = true;
        }
    }
    if (positiveFlag && negativeFlag) {
        int j = len - 1;
        bool flag = false;
        for (int i = 0; i < len; i++) {
            out[i] = in[j];
            if (flag) {
                j++;
            } else {
                j--;
            }
            if (j == firstNegativeIndex - 1 && !flag) {
                j = 0;
                flag = true;
            }
        }
        return true;
    } else if (!positiveFlag) {
        for (int i = len - 1, j = 0; i >= 0; i--, j++) {
            out[j] = in[i];
        }
        return true;
    }
    return false;
}



std::vector<double> radixSortSeq(const std::vector<double>& data) {
    int len = static_cast<int>(data.size());
    std::vector<double> in = data;
    std::vector<double> out(data.size());
    for (int i = 0; i < 4; i++) {
        countSort(in.data(), out.data(), len, 2 * i);
        countSort(out.data(), in.data(), len, 2 * i + 1);
    }
    if (!countSortFinalStep(in.data(), out.data(), len)) {
        in.swap(out);
    }
    return out;
}

std::vector<std::vector<double>> splitVector(const std::vector<double>& data, int numParts) {
    std::vector<std::vector<double>> resultVec;
    std::vector<double> tmp;

    if (numParts < 2 || data.size() < numParts) {
        resultVec.push_back(data);
        return resultVec;
    }

    int i = 0;
    int part = (data.size() / numParts) + 1;
    for (i = 0; i < numParts - 1; i++) {
        for (int j = i * part; j < (i + 1) * part; j++) {
            tmp.push_back(data[j]);
        }
        resultVec.push_back(tmp);
        tmp.clear();
    }
    for (int j = i * part; j < data.size(); j++) {
        tmp.push_back(data[j]);
    }
    resultVec.push_back(tmp);
    return resultVec;
}

std::vector<double> radixSortParallOmp(const std::vector<double>& data, int numParts) {
    std::vector<std::vector<double>> vectorsForParallel = splitVector(data, numParts);
    std::vector<double> finalResult;
    omp_set_num_threads(vectorsForParallel.size());
    // auto start1 = std::chrono::high_resolution_clock::now();
#pragma omp parallel for
    for (int i = 0; i < vectorsForParallel.size(); ++i) {
        vectorsForParallel[i] = radixSortSeq(vectorsForParallel[i]);
    }

    // auto stop1 = std::chrono::high_resolution_clock::now();
    // auto duration1 = std::chrono::duration_cast<std::chrono::microseconds>(stop1 - start1);
    // std::cout << "duration_omp: " << duration1.count() << '\n';
    for (int i = 0; i < vectorsForParallel.size(); i++) {
        finalResult = merge(finalResult, vectorsForParallel[i]);
    }
    // printVector(finalResult);
    return finalResult;
}

void printVector(const std::vector<double>& vec) {
    for (int i = 0; i < vec.size(); i++) {
        std::cout << vec[i] << " ";
    }
    std::cout << std::endl;
}

std::vector<double> getRandomVector(int  size, int leftBound, int rightBound) {
    std::vector<double> vec(size);
    std::random_device dev;
    std::mt19937 gen(dev());
    std::uniform_real_distribution<double> buf(leftBound, rightBound);
    for (int i = 0; i < size; i++) {
        vec[i] = buf(gen);
    }
    return vec;
}

std::vector<double> merge(const std::vector<double>& arr1,
    const std::vector<double>& arr2) {
    int len1 = arr1.size();
    int len2 = arr2.size();
    std::vector<double> out;
    out.resize(len1 + len2);
    int indexFirst = 0, indexSecond = 0;
    for (int i = 0; i < len1 + len2; i++) {
        if (indexFirst == len1) {
            out[i] = arr2[indexSecond];
            indexSecond++;
            continue;
        }
        if (indexSecond == len2) {
            out[i] = arr1[indexFirst];
            indexFirst++;
            continue;
        }
        if (arr1[indexFirst] > arr2[indexSecond]) {
            out[i] = arr2[indexSecond];
            indexSecond++;
        } else if (arr1[indexFirst] <= arr2[indexSecond]) {
            out[i] = arr1[indexFirst];
            indexFirst++;
        }
    }
    return out;
}

void countSort(double* in, double* out, int len, int exp) {
    unsigned char* buf = reinterpret_cast<unsigned char*>(in);
    int count[256] = { 0 };
    for (int i = 0; i < len; i++) {
        count[buf[8 * i + exp]]++;
    }
    /*for (int i = 0; i < 256; i++) {
        std::cout << count[i] << " ";
    }
    std::cout << std::endl;*/
    int sum = 0;
    for (int i = 0; i < 256; i++) {
        int temp = count[i];
        count[i] = sum;
        sum += temp;
    }
    /*for (int i = 0; i < 256; i++) {
        std::cout << count[i] << " ";
    }
    std::cout << std::endl;*/
    for (int i = 0; i < len; i++) {
        out[count[buf[8 * i + exp]]] = in[i];
        count[buf[8 * i + exp]]++;
    }
}

bool countSortFinalStep(double* in, double* out, int len) {
    int firstNegativeIndex = -1;
    int firstPositiveIndex = -1;
    bool positiveFlag = false, negativeFlag = false;
    for (int i = 0; i < len; i++) {
        if (positiveFlag && negativeFlag) {
            break;
        }
        if (in[i] < 0 && !negativeFlag) {
            firstNegativeIndex = i;
            negativeFlag = true;
        }
        if (in[i] > 0 && !positiveFlag) {
            firstPositiveIndex = i;
            positiveFlag = true;
        }
    }
    if (positiveFlag && negativeFlag) {
        int j = len - 1;
        bool flag = false;
        for (int i = 0; i < len; i++) {
            out[i] = in[j];
            if (flag) {
                j++;
            } else {
                j--;
            }
            if (j == firstNegativeIndex - 1 && !flag) {
                j = 0;
                flag = true;
            }
        }
        return true;
    } else if (!positiveFlag) {
        for (int i = len - 1, j = 0; i >= 0; i--, j++) {
            out[j] = in[i];
        }
        return true;
    }
    return false;
}



std::vector<double> radixSortSeq(const std::vector<double>& data) {
    int len = static_cast<int>(data.size());
    std::vector<double> in = data;
    std::vector<double> out(data.size());
    for (int i = 0; i < 4; i++) {
        countSort(in.data(), out.data(), len, 2 * i);
        countSort(out.data(), in.data(), len, 2 * i + 1);
    }
    if (!countSortFinalStep(in.data(), out.data(), len)) {
        in.swap(out);
    }
    return out;
}

std::vector<std::vector<double>> splitVector(const std::vector<double>& data, int numParts) {
    std::vector<std::vector<double>> resultVec;
    std::vector<double> tmp;

    if (numParts < 2 || data.size() < numParts) {
        resultVec.push_back(data);
        return resultVec;
    }

    int i = 0;
    int part = (data.size() / numParts) + 1;
    for (i = 0; i < numParts - 1; i++) {
        for (int j = i * part; j < (i + 1) * part; j++) {
            tmp.push_back(data[j]);
        }
        resultVec.push_back(tmp);
        tmp.clear();
    }
    for (int j = i * part; j < data.size(); j++) {
        tmp.push_back(data[j]);
    }
    resultVec.push_back(tmp);
    return resultVec;
}

std::vector<double> radixSortParallOmp(const std::vector<double>& data, int numParts) {
    std::vector<std::vector<double>> vectorsForParallel = splitVector(data, numParts);
    std::vector<double> finalResult;
    // auto start1 = std::chrono::high_resolution_clock::now();

    tbb::global_control gc(tbb::global_control::max_allowed_parallelism, vectorsForParallel.size());
    tbb::parallel_for(tbb::blocked_range<double>(0, vectorsForParallel.size()),
        [&](const tbb::blocked_range<double>& r) {
        for (int i = r.begin(); i < r.end(); ++i) {
            vectorsForParallel[i] = radixSortSeq(vectorsForParallel[i]);
        }
        });

    // auto stop1 = std::chrono::high_resolution_clock::now();
    // auto duration1 = std::chrono::duration_cast<std::chrono::microseconds>(stop1 - start1);
    // std::cout << "duration_tbb: " << duration1.count() << '\n';
    for (int i = 0; i < vectorsForParallel.size(); i++) {
        finalResult = merge(finalResult, vectorsForParallel[i]);
    }
    // printVector(finalResult);
    return finalResult;
}

void printVector(const std::vector<double>& vec) {
    for (int i = 0; i < vec.size(); i++) {
        std::cout << vec[i] << " ";
    }
    std::cout << std::endl;
}

std::vector<double> getRandomVector(int  size, int leftBound, int rightBound) {
    std::vector<double> vec(size);
    std::random_device dev;
    std::mt19937 gen(dev());
    std::uniform_real_distribution<double> buf(leftBound, rightBound);
    for (int i = 0; i < size; i++) {
        vec[i] = buf(gen);
    }
    return vec;
}

std::vector<double> merge(const std::vector<double>& arr1,
    const std::vector<double>& arr2) {
    int len1 = arr1.size();
    int len2 = arr2.size();
    std::vector<double> out;
    out.resize(len1 + len2);
    int indexFirst = 0, indexSecond = 0;
    for (int i = 0; i < len1 + len2; i++) {
        if (indexFirst == len1) {
            out[i] = arr2[indexSecond];
            indexSecond++;
            continue;
        }
        if (indexSecond == len2) {
            out[i] = arr1[indexFirst];
            indexFirst++;
            continue;
        }
        if (arr1[indexFirst] > arr2[indexSecond]) {
            out[i] = arr2[indexSecond];
            indexSecond++;
        } else if (arr1[indexFirst] <= arr2[indexSecond]) {
            out[i] = arr1[indexFirst];
            indexFirst++;
        }
    }
    return out;
}

void countSort(double* in, double* out, int len, int exp) {
    unsigned char* buf = reinterpret_cast<unsigned char*>(in);
    int count[256] = { 0 };
    for (int i = 0; i < len; i++) {
        count[buf[8 * i + exp]]++;
    }
    /*for (int i = 0; i < 256; i++) {
        std::cout << count[i] << " ";
    }
    std::cout << std::endl;*/
    int sum = 0;
    for (int i = 0; i < 256; i++) {
        int temp = count[i];
        count[i] = sum;
        sum += temp;
    }
    /*for (int i = 0; i < 256; i++) {
        std::cout << count[i] << " ";
    }
    std::cout << std::endl;*/
    for (int i = 0; i < len; i++) {
        out[count[buf[8 * i + exp]]] = in[i];
        count[buf[8 * i + exp]]++;
    }
}

bool countSortFinalStep(double* in, double* out, int len) {
    int firstNegativeIndex = -1;
    int firstPositiveIndex = -1;
    bool positiveFlag = false, negativeFlag = false;
    for (int i = 0; i < len; i++) {
        if (positiveFlag && negativeFlag) {
            break;
        }
        if (in[i] < 0 && !negativeFlag) {
            firstNegativeIndex = i;
            negativeFlag = true;
        }
        if (in[i] > 0 && !positiveFlag) {
            firstPositiveIndex = i;
            positiveFlag = true;
        }
    }
    if (positiveFlag && negativeFlag) {
        int j = len - 1;
        bool flag = false;
        for (int i = 0; i < len; i++) {
            out[i] = in[j];
            if (flag) {
                j++;
            } else {
                j--;
            }
            if (j == firstNegativeIndex - 1 && !flag) {
                j = 0;
                flag = true;
            }
        }
        return true;
    } else if (!positiveFlag) {
        for (int i = len - 1, j = 0; i >= 0; i--, j++) {
            out[j] = in[i];
        }
        return true;
    }
    return false;
}



std::vector<double> radixSortSeq(const std::vector<double>& data) {
    int len = static_cast<int>(data.size());
    std::vector<double> in = data;
    std::vector<double> out(data.size());
    for (int i = 0; i < 4; i++) {
        countSort(in.data(), out.data(), len, 2 * i);
        countSort(out.data(), in.data(), len, 2 * i + 1);
    }
    if (!countSortFinalStep(in.data(), out.data(), len)) {
        in.swap(out);
    }
    return out;
}

std::vector<std::vector<double>> splitVector(const std::vector<double>& data, int numParts) {
    std::vector<std::vector<double>> resultVec;
    std::vector<double> tmp;

    if (numParts < 2 || data.size() < numParts) {
        resultVec.push_back(data);
        return resultVec;
    }

    int i = 0;
    int part = (data.size() / numParts) + 1;
    for (i = 0; i < numParts - 1; i++) {
        for (int j = i * part; j < (i + 1) * part; j++) {
            tmp.push_back(data[j]);
        }
        resultVec.push_back(tmp);
        tmp.clear();
    }
    for (int j = i * part; j < data.size(); j++) {
        tmp.push_back(data[j]);
    }
    resultVec.push_back(tmp);
    return resultVec;
}

std::vector<double> radixSortParallOmp(const std::vector<double>& data, int numParts) {
    std::vector<std::vector<double>> vectorsForParallel = splitVector(data, numParts);
    std::vector<double> finalResult;
     // auto start1 = std::chrono::high_resolution_clock::now();

    std::vector<std::thread> threads;
    for (int i = 0; i < vectorsForParallel.size(); i++) {
        threads.emplace_back([i, &vectorsForParallel]() {
            vectorsForParallel[i] = radixSortSeq(vectorsForParallel[i]);
            });
    }

    for (auto& thread : threads) {
        thread.join();
    }

     // auto stop1 = std::chrono::high_resolution_clock::now();
     // auto duration1 = std::chrono::duration_cast<std::chrono::microseconds>(stop1 - start1);
     // std::cout << "duration_std: " << duration1.count() << '\n';
    for (int i = 0; i < vectorsForParallel.size(); i++) {
        finalResult = merge(finalResult, vectorsForParallel[i]);
    }
    // printVector(finalResult);
    return finalResult;
}
\end{verbatim}

\end{document}