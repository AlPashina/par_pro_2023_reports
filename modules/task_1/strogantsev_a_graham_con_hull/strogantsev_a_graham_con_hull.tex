\documentclass[14pt, a4paper]{extarticle}
\usepackage[utf8]{inputenc}
\usepackage[T2A]{fontenc}
\usepackage[a4paper,
left=2cm,
right=2cm,
top=2cm,
bottom=2cm]{geometry}
\usepackage{color}
\usepackage{mathtools}
\usepackage{listings}
\usepackage{graphicx}
\usepackage{tocloft}
\usepackage{indentfirst}
\usepackage{enumitem}
\usepackage[russian]{babel}

\setlength{\parindent}{0.8cm}
\setlength{\parskip}{0.4cm}
\renewcommand{\contentsname}{}
\renewcommand{\cftsecleader}{\cftdotfill{\cftdotsep}}
\definecolor{dkgreen}{rgb}{0,0.6,0}
\definecolor{gray}{rgb}{0.5,0.5,0.5}
\definecolor{mauve}{rgb}{0.58,0,0.82}

\lstset{frame=none,
  language=C++,
  aboveskip=3mm,
  belowskip=3mm,
  showstringspaces=false,
  columns=flexible,
  basicstyle={\small\ttfamily},
  numbers=none,
  numberstyle=\tiny\color{gray},
  keywordstyle=\color{blue},
  commentstyle=\color{dkgreen},
  stringstyle=\color{mauve},
  breaklines=true,
  breakatwhitespace=true,
  tabsize=4
}

\title{}
\author{}
\date{}

\begin{document}
  \begin{titlepage}
    \begin{center}
      {\bfseries МИНИСТЕРСТВО НАУКИ И ВЫСШЕГО ОБРАЗОВАНИЯ \\
        РОССИЙСКОЙ ФЕДЕРАЦИИ}
      \\
      Федеральное государственное автономное образовательное учреждение высшего образования
      \\
      {\bfseries «Национальный исследовательский Нижегородский государственный университет им. Н.И. Лобачевского»\\(ННГУ)
        \\Институт информационных технологий, математики и механики} \\
    \end{center}

    \vspace{8em}

    \begin{center}
      ОТЧЕТ \\ по лабораторной работе \\
      «Применение библиотек OpenMP и TBB для параллелизации решения задачи о построении выпуклой оболочки – проход Грэхема.»
    \end{center}

    \vspace{5em}


    \begin{flushright}
      {\bfseries Выполнил:} студент группы\\382006-2\\Строганцев А.М. \underline{\hspace{3cm}} \linebreak\linebreak\linebreak
      {\bfseries Проверил:} младший научный\\сотрудник\\Нестеров А.Ю. \underline{\hspace{3cm}} 
    \end{flushright}


    \vspace{\fill}

    \begin{center}
      Нижний Новгород\\2023
    \end{center}

  \end{titlepage}

  \tableofcontents
  \thispagestyle{empty}
  \newpage

  \pagestyle{plain}
  \setcounter{page}{3}

  \section{Введение}
     В ходе выполнения лабораторной работы будет рассмотрено применение различных методов параллелизации для решения задачи о построении выпуклой оболочки – проход Грэхема. В частности такие библиотеки как: OpenMP и TBB.

    Проход Грэхема – это один из наиболее известных алгоритмов для построения выпуклой оболочки множества точек на плоскости. В его основе лежит идея последовательного выбора точек, которые образуют выпуклую оболочку. Однако, проход Грэхема имеет линейную сложность, что не позволяет эффективно обрабатывать большие объемы данных.
    
    Для повышения эффективности можно использовать методы параллелизации, которые позволяют распределить вычисления между несколькими потоками.
    
    Цель данной работы – исследовать возможности и эффективность двух библиотек параллелизации для решения задачи о построении выпуклой оболочки – проход Грэхема.
  \newpage

  \section{Постановка задачи}
  Следует реализовать 2 рабочих варианта библиотеки/приложения, позволяющих интегрировать и использовать параллельную версию алгоритма построения выпуклой оболочки Грэхема.
    \begin{itemize}
      \item Программы должны быть созданы с помощью технологии OpenMP и TBB на языке C++.
        \item В каждом варианте должна существовать функция принимающая на вход множество точек и возращающая выпуклую оболочку, созданную на основе исходного множества.
        \item Корректным результатом работы является нахождение уникальной последовательности точек, представляющих выпуклую оболочку, образованную проходом против часовой стрелки, начиная с точки с наименьшей ординатой и абсциссой.
        \item Также для установления некоторых закономерностей и проделывания выводов необходимо замерять время работы каждой версии алгоритма.
    \end{itemize}
  Далеее подведем итоги, где сравним реализации алгоритма между собой.
  \newpage

  \section{Описание алгоритма}
  \begin{enumerate}
      \item Находим самую нижнюю точку множества, называемую начальной точкой. Если таких точек несколько, то выбираем из них самую левую.

        \item Сортируем все точки по углу, который они образуют с горизонтальной осью, проходящей через начальную точку.

        \item Создаем стек и кладем в него первые две точки.

        \item Просматриваем оставшиеся точки, начиная с третьей. Для каждой точки выполняем следующие действия:
        
        \begin{enumerate}[label*=\arabic*.]
            \item Пока точка является внутренней для выпуклой оболочки, удаляем последнюю точку из стека. Это делается до тех пор, пока текущая точка не станет внешней для выпуклой оболочки.
            \item Кладем текущую точку в стек.
        \end{enumerate}

        \item В конце алгоритма в стеке останутся только точки, образующие выпуклую оболочку множества.
  \end{enumerate}
  \newpage

  \section{Описание схемы распараллеливания}
  Параллельная модификация заключается в разделении основного массива точек полученного после 2-го шага алгоритма на несколько частей для последующей обработки в нескольких потоках. Следующие шаги остаются неизменными, за исключением того, что выполняются они не на всем массиве точек, а только на части, выделенной единице параллелизма.

  \newpage

  \section{Описание схемы параллелизации}

    \subsection{OpenMP}
    \textbf{Метод параллелизации:}\\
  \begin{enumerate}
    \item Определить количество задействованных потоков.
    \item Определить идентификатор текущего потока.
    \item Вычислить левую и правую границу в исходном наборе точек для обработки в текущем потоке.
    \item Выполнить алгоритм построения выпуклой оболочки.
    \item В критической секции объединить результаты работы всех потоков.
    \item Выполнить основной алгоритм на результирующем массиве точек.
  \end{enumerate}

   \subsection{TBB}
   Изначально объявлен класс ConvexHullTbbMediator, который реализует:
   \begin{itemize}
       \item Дефолтный конструктор по умолчанию
       \item Пустой конструктор копирования с маркерным вторым параметром. 
       \item Оператор "invoke"\ , принимающий диапазон точек, который выполняет основной алгоритм на заданном диапазоне и сохраняет результат в поле класса.
       \item Метод "join"\ , который объединяет свой стек со стеком переданного аргументом объекта.
   \end{itemize}
    \textbf{Метод параллелизации:}
  \begin{enumerate}
    \item Создание инстанса класса ConvexHullTbbMediator.
        \item Вызов функции tbb::parallel\_for с диапазоном исходных точек и объектом ConvexHullTbbMediator.
        \item Выполнить основной алгоритм на результирующем массиве точек.
  \end{enumerate}
 

  \newpage

  \section{Результаты экспериментов}
  Каждый эксперимент заключается в определении размера множества входных данных, изначально массив заполняется 4 точками, которые являются выпуклой оболочкой, далее вектор дополняется случайной точкой, которая не лежит на оболочке, до определенного размера. После чего запускает основной алгоритм, подсчитывается его время выполнения, данные заносятся в таблицу (выбирается лучший результат из 3х замеров).
 
  \noindent\textbf{Результаты измерения времени работы (в миллисекундах):}\\\\
    \begin{tabular}{|c | c | c | c | c | c |} 
      \hline
      Технология / Размер входных данных & $10^3$ & $10^4$ & $10^5$ & $10^6$ & $10^7$ \\
      \hline
      OpenMP & 2 & 36  & 200 & 1450 & 11924 \\ 
      \hline
      TBB & 3 & 41  & 186 & 1382 & 10703 \\ 
      \hline
    \end{tabular}

  \newpage

  \section{Вывод}
  Основываясь на полученных в результате экспериментов данных, можно сделать вывод: обе библиотеки показали примерно равные результаты в приведенных экспериментах, что говорит об их схожей степени эффективности.

  \newpage

  \section{Заключение}
  В ходе выполнения работы были изучены и успешно реализованы 2 варианта параллельной версии алгоритма построения выпуклой оболочки Греэхема.
 
  В ходе экспериментов удалось выяснить, что обе реализации имеют примерно равное время выполнения.
 
    Следовательно, выбор библиотеки может быть сделан на основе удобства интеграции конкретного функционала в пользовательское приложение. В то же время возможно использовать оба механизма для наибольшей эффективности в эксплуатации. Так OpenMP больше подходит для параллелизации независимых вычислений внутри цикла, а TBB предоставляет удобный интерфейс для редуцирования результатов работы условных единиц параллелизации.
 
  \newpage

    \section{Литература}
    
    \begin{enumerate}
        \item Справочник по библиотекам OpenMP // Справочник по библиотекам OpenMP URL: \\https://learn.microsoft.com/ru-ru/cpp/parallel/openmp/reference/openmp-library-reference?view=msvc-170 (дата обращения: 04.05.2023).
        \item Threading Building Blocks // Threading Building Blocks URL: https://oneapi-src.github.io/oneTBB/ (дата обращения: 04.05.2023).
        \item Построение минимальных выпуклых оболочек // habr.com URL:\\ https://habr.com/ru/articles/144921/ (дата обращения: 04.05.2023).
        \item  Алгоритм Грэхема // wikipedia.org URL:\\ https://ru.wikipedia.org/wiki/Алгоритм\_Грэхема (дата обращения: 04.05.2023). 
    \end{enumerate}

  \newpage

  \section{Приложение}

  \subsection{Общая часть}
    \subsubsection{main.cpp}
  \begin{lstlisting}
#include <gtest/gtest.h>
#include "./graham_con_hull_omp.h"

TEST(ConvexHullOMP, EmptyVector) {
    std::vector<Point> points;
    auto hull = constructConvexHull(points);
    EXPECT_TRUE(hull.empty());
}

TEST(ConvexHullOMP, ThreePoints) {
    std::vector<Point> points{ {0, 0}, {1, 1}, {0, 1} };
    auto hull = constructConvexHull(points);
    EXPECT_EQ(hull.size(), 3);
    EXPECT_EQ(hull[0], Point(0, 0));
    EXPECT_EQ(hull[1], Point(1, 1));
    EXPECT_EQ(hull[2], Point(0, 1));
}

TEST(ConvexHullOMP, FourPoints) {
    std::vector<Point> points{ {0, 0}, {1, 1}, {0, 1}, {1, 0} };
    auto hull = constructConvexHull(points);
    EXPECT_EQ(hull.size(), 4);
    EXPECT_EQ(hull[0], Point(0, 0));
    EXPECT_EQ(hull[1], Point(1, 0));
    EXPECT_EQ(hull[2], Point(1, 1));
    EXPECT_EQ(hull[3], Point(0, 1));
}

TEST(ConvexHullOMP, TenPoints) {
    std::vector<Point> points{
        {0, 0}, {10, 10}, {0, 10}, {10, 0}, {11, 11},
        {5, 5}, {3, 1}, {2, 4}, {7, 3}, {11, 11}
    };
    auto hull = constructConvexHull(points);
    EXPECT_EQ(hull.size(), 4);
    EXPECT_EQ(hull[0], Point(0, 0));
    EXPECT_EQ(hull[1], Point(10, 0));
    EXPECT_EQ(hull[2], Point(11, 11));
    EXPECT_EQ(hull[3], Point(0, 10));
}

TEST(ConvexHullOMP, PointsInALine) {
    std::vector<Point> points{ {0, 0}, {1, 0}, {2, 0}, {3, 0} };
    auto hull = constructConvexHull(points);
    EXPECT_EQ(hull.size(), 2);
    EXPECT_EQ(hull[0], Point(0, 0));
    EXPECT_EQ(hull[1], Point(3, 0));
}

int main(int argc, char **argv) {
    ::testing::InitGoogleTest(&argc, argv);
    return RUN_ALL_TESTS();
}
  \end{lstlisting}
  \newpage
  \subsubsection{graham\_convex\_hull.h}
  \begin{lstlisting}
#include <vector>

struct Point {
    int x;
    int y;

    Point(int x, int y) : x(x), y(y) {}

    bool operator==(const Point& other) const {
        return x == other.x && y == other.y;
    }
};

enum class Orientation {
    COLLINEAR,
    LEFT,
    RIGHT,
};

Orientation getOrientation(const Point& p1, const Point& p2, const Point& anchor);

bool comparePointsOnAnchor(const Point& p1, const Point& p2, const Point& anchor);

std::vector<Point> constructConvexHull(const std::vector<Point>& points);
  \end{lstlisting}
  \newpage
    \subsection{OpenMP - graham\_convex\_hull.cpp}
  \begin{lstlisting}
#include <omp.h>
#include <iostream>
#include <algorithm>
#include <iterator>
#include "../../../modules/task_2/strogantsev_a_graham_con_hull_omp/graham_con_hull_omp.h"


Orientation getOrientation(const Point& p1, const Point& p2, const Point& anchor) {
    int value = (p2.y - p1.y) * (anchor.x - p2.x) - (p2.x - p1.x) * (anchor.y - p2.y);
    if (value == 0) return Orientation::COLLINEAR;
    if (value > 0) return Orientation::LEFT;
    return Orientation::RIGHT;
}

bool comparePointsOnAnchor(const Point& p1, const Point& p2, const Point& anchor) {
    auto orientation = getOrientation(anchor, p1, p2);
    if (orientation == Orientation::COLLINEAR)
        return ((p1.x - anchor.x) * (p1.x - anchor.x) + (p1.y - anchor.y) * (p1.y - anchor.y))
        < ((p2.x - anchor.x) * (p2.x - anchor.x) + (p2.y - anchor.y) * (p2.y - anchor.y));
    return orientation == Orientation::RIGHT;
}

std::vector<Point> constructConvexHullSeq(const std::vector<Point>& points) {
    if (points.size() < 3) return points;

    auto pointsCopy = std::vector<Point>(points.begin(), points.end());

    Point anchor = *std::min_element(pointsCopy.begin(), pointsCopy.end(), [](const Point& p1, const Point& p2) {
        if (p1.y == p2.y) return p1.x < p2.x;
        return p1.y < p2.y;
    });

    std::sort(pointsCopy.begin(), pointsCopy.end(), [&](const Point& p1, const Point& p2) {
        return comparePointsOnAnchor(p1, p2, anchor);
    });

    std::vector<Point> stack;
    stack.push_back(pointsCopy[0]);
    stack.push_back(pointsCopy[1]);
    stack.push_back(pointsCopy[2]);

    for (int i = 3; i < pointsCopy.size(); i++) {
        while (stack.size() > 1 &&
            getOrientation(stack[stack.size() - 2], stack.back(), pointsCopy[i]) != Orientation::RIGHT)
            stack.pop_back();
        stack.push_back(pointsCopy[i]);
    }
    return stack;
}

std::vector<Point> constructConvexHull(const std::vector<Point>& points) {
    if (points.empty()) return points;

    Point anchor = *std::min_element(points.begin(), points.end(), [](const Point& p1, const Point& p2) {
        if (p1.y == p2.y) return p1.x < p2.x;
        return p1.y < p2.y;
    });

    auto pointsCopy = points;
    std::sort(pointsCopy.begin(), pointsCopy.end(), [&](const Point& p1, const Point& p2) {
        return comparePointsOnAnchor(p1, p2, anchor);
    });

    if (points.size() < 3) return pointsCopy;

    std::vector<Point> stack;
    stack.push_back(pointsCopy[0]);
    stack.push_back(pointsCopy[1]);

#pragma omp parallel
    {
        const int threadId = omp_get_thread_num();
        const int threadCount = omp_get_num_threads();
        const int step = (points.size() - 2) / threadCount + (((points.size() - 2) % threadCount) != 0);
        std::vector<Point> localStack;
        for (int i = 2 + threadId * step;
            i < std::min(static_cast<int>(pointsCopy.size()), 2 + step * (threadId + 1)); i++) {
            while (localStack.size() > 1 &&
                getOrientation(localStack[localStack.size() - 2], localStack.back(), pointsCopy[i]) !=
                Orientation::RIGHT)
                localStack.pop_back();
            localStack.push_back(pointsCopy[i]);
        }
#pragma omp critical
        {
            std::copy(localStack.begin(), localStack.end(), std::back_inserter(stack));
        }
    }
    return constructConvexHullSeq(stack);
}
  \end{lstlisting}
  \newpage
    \subsection{TBB - graham\_convex\_hull.cpp}
  \begin{lstlisting}
 #include <tbb/tbb.h>
#include <iostream>
#include <algorithm>
#include <iterator>
#include "../../../modules/task_3/strogantsev_a_graham_con_hull_tbb/graham_con_hull_tbb.h"


enum class Orientation {
    COLLINEAR,
    LEFT,
    RIGHT,
};

Orientation getOrientation(const Point& p1, const Point& p2, const Point& anchor) {
    int value = (p2.y - p1.y) * (anchor.x - p2.x) - (p2.x - p1.x) * (anchor.y - p2.y);
    if (value == 0) return Orientation::COLLINEAR;
    if (value > 0) return Orientation::LEFT;
    return Orientation::RIGHT;
}

bool comparePointsOnAnchor(const Point& p1, const Point& p2, const Point& anchor) {
    auto orientation = getOrientation(anchor, p1, p2);
    if (orientation == Orientation::COLLINEAR)
        return ((p1.x - anchor.x) * (p1.x - anchor.x) + (p1.y - anchor.y) * (p1.y - anchor.y))
        < ((p2.x - anchor.x) * (p2.x - anchor.x) + (p2.y - anchor.y) * (p2.y - anchor.y));
    return orientation == Orientation::RIGHT;
}

std::vector<Point> constructConvexHullSeq(const std::vector<Point>& points) {
    if (points.size() < 3) return points;

    auto pointsCopy = std::vector<Point>(points.begin(), points.end());

    Point anchor = *std::min_element(pointsCopy.begin(), pointsCopy.end(), [](const Point& p1, const Point& p2) {
        if (p1.y == p2.y) return p1.x < p2.x;
        return p1.y < p2.y;
        });

    std::sort(pointsCopy.begin(), pointsCopy.end(), [&](const Point& p1, const Point& p2) {
        return comparePointsOnAnchor(p1, p2, anchor);
        });

    std::vector<Point> stack;
    stack.push_back(pointsCopy[0]);
    stack.push_back(pointsCopy[1]);
    stack.push_back(pointsCopy[2]);

    for (int i = 3; i < pointsCopy.size(); i++) {
        while (stack.size() > 1 &&
            getOrientation(stack[stack.size() - 2], stack.back(), pointsCopy[i]) != Orientation::RIGHT)
            stack.pop_back();
        stack.push_back(pointsCopy[i]);
    }
    return stack;
}

struct ConvexHullTbbMediator {
    mutable std::vector<Point> stack;

    ConvexHullTbbMediator() = default;
    ConvexHullTbbMediator(const ConvexHullTbbMediator& s, tbb::split) {}

    void operator()(const tbb::blocked_range<std::vector<Point>::iterator>& points) {
        for (const Point& point : points) {
            while (stack.size() > 1 &&
                getOrientation(stack[stack.size() - 2], stack.back(), point) != Orientation::RIGHT)
                stack.pop_back();
            stack.push_back(point);
        }
    }
    void join(const ConvexHullTbbMediator& r) {
        std::copy(r.stack.begin(), r.stack.end(), std::back_inserter(stack));
    }
};

std::vector<Point> constructConvexHull(const std::vector<Point>& points) {
    if (points.empty()) return points;

    Point anchor = *std::min_element(points.begin(), points.end(), [](const Point& p1, const Point& p2) {
        if (p1.y == p2.y) return p1.x < p2.x;
        return p1.y < p2.y;
        });

    auto pointsCopy = points;
    std::sort(pointsCopy.begin(), pointsCopy.end(), [&](const Point& p1, const Point& p2) {
        return comparePointsOnAnchor(p1, p2, anchor);
        });

    if (points.size() < 3) return pointsCopy;

    ConvexHullTbbMediator mediator;
    tbb::parallel_reduce(
        tbb::blocked_range<std::vector<Point>::iterator>(pointsCopy.begin(), pointsCopy.end()),
        mediator);

    return constructConvexHullSeq(mediator.stack);
}
    \end{lstlisting}

\end{document}